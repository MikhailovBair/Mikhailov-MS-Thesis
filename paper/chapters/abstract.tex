The phylogenetic relationship between tomistomas, gavials, crocodiles, and alligators remains unresolved due to conflicting morphological and molecular evidence. This study introduces a machine learning framework to analyze brain endocasts derived from CT scans, aiming to resolve these evolutionary uncertainties. We first segmented brain endocasts from crocodilian cranial scans using a 2D U-Net architecture, achieving a mean Dice score of 0.75. 
A ResNet classifier WILL BE trained on segmented endocasts to distinguish crocodiles from alligators. Applying this model to tomistomas and gavials revealed a closer neuroanatomical affinity to crocodiles, supporting the shared ancestry hypothesis. Grad-CAM visualizations WILL BE used to identify critical discriminative features.
Results demonstrate the potential of explainable deep learning to address phylogenetic controversies, offering a scalable, data-driven alternative to subjective morphological comparisons. This work bridges computational radiology and evolutionary biology, providing a template for quantitative neuroanatomical phenotyping in extinct and extant species.